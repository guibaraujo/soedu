\documentclass{article}
\usepackage{graphicx}
%encoding
%--------------------------------------
\usepackage[T1]{fontenc}
\usepackage[utf8]{inputenc}
%--------------------------------------
 
%Portuguese-specific commands
%--------------------------------------
\usepackage[portuguese]{babel}
%--------------------------------------

\begin{document}

\title{SO.EDU MANUAL}
%\author{Author's Name}

\maketitle

%\begin{abstract}
%This is abstract text: This simple document shows very basic features of
%\LaTeX{}.
%\end{abstract}

\section{Cenários de Teste}

%\begin{equation}
%    \label{simple_equation}
%    \alpha = \sqrt{ \beta }
%\end{equation}

\subsection{Algoritmos de Escalonamento}

\begin{itemize}
	\item FIFO (\textit{First In First Out})
	\item RR (\textit{Round Robin})
	\item SJF (\textit{Shortest Job First})
	\item EDF (\textit{Earliest Deadline First})
\end{itemize}

\subsection{Algorítimos de Substituição de Páginas}

\begin{itemize}
	\item FIFO
		\begin{table}[!htbp]
		\centering
		\label{undefined}
		\begin{tabular}{|c|c|c|c|c|}
		\hline
		PID & Tempo de Chegada & Tempo de Execução & Número de Páginas & Deadline \\ \hline
   		1&    1&    4&    10& 0\\ \hline
   		2&    4&   4&    10& 0\\ \hline
   		3&    5&   4&    10& 0	\\ \hline
   		4&    8&   4&    10& 0	\\ \hline
   		5&    8&   2&    10& 0	\\ \hline
   		6&    10&   2&    10& 0\\ \hline
   		7&    11&   2&    10& 0\\ \hline
   		8&    15&   2&    10& 0\\ \hline
   		9&    16&   2&    10& 0\\ \hline
   		10&   20&   2&   10& 0\\ \hline
		\end{tabular}
		\end{table}

	\item LRU	
		\begin{table}[!htbp]
		\centering
		\label{undefined}
		\begin{tabular}{|c|c|c|c|c|}
		\hline
		PID & Tempo de Chegada & Tempo de Execução & Número de Páginas & Deadline \\ \hline
   		1&     2& 4&    10& 0\\ \hline
   		2&    2& 4&    10& 0\\ \hline
   		3&    10&   4&    10& 0	\\ \hline
   		4&    10&   4&    10& 0	\\ \hline
   		5&    10&   2&    10& 0	\\ \hline
   		6&    10&   2&    10& 0	\\ \hline
		\end{tabular}
		\end{table}
\end{itemize}
%\begin{figure}
%  \centering
%    \includegraphics[width=3.0in]{myfigure}
%    \caption{Simulation Results}
%    \label{simulationfigure}
%\end{figure}

\end{document}